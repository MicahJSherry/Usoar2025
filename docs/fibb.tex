\documentclass{article}


\usepackage[english]{babel}
\usepackage{listings}
\usepackage[letterpaper, top=2.5cm,bottom=2.5cm,left=2.5cm,right=2.5cm,marginparwidth=1.75cm]{geometry}

\usepackage{amsmath,amsthm,amssymb}
\usepackage{bbold}
\usepackage{graphicx}
\usepackage{listings}
\usepackage{multicol}
\usepackage{setspace}
\usepackage{enumitem}


%common set symbols 
\newcommand{\Z}{\mathbb{Z}}
\newcommand{\N}{\mathbb{N}}
\newcommand{\R}{\mathbb{R}}
\newcommand{\Q}{\mathbb{Q}}
\newcommand{\C}{\mathbb{C}}
\newcommand{\B}{\mathbb{B}}
\newcommand{\nullset}{\varnothing}
\newcommand{\powerset}[1]{\mathcal{P} ({#1})}
\newcommand{\ol}[1]{\overline{#1}}

\newcommand{\divides}{\mid}
\newcommand{\notdivides}{\nmid}

\newcommand{\st}{\text{ such that }}
\newcommand{\pmi}{\text{ principle of mathematical induction }}

\newcommand{\nitem}[1] % sets enumerate to a specific number {arg1}
{
	\setcounter{enumi}{#1}
	\addtocounter{enumi}{-1}
	\item
}
\usepackage{multicol}
\newtheorem{theorem}{Theorem}
\newtheorem{lemma}{Lemma}
\title{ }

\begin{document}

	\maketitle
    \section*{Definitions}
    
    Let $f_n(x) = \sum_{i=0}^{n}x^i$.\\
    Let $G_n$ be the set of all $g(x)$ such that $g(x)= \sum_{i=1}^{n} a_i x^n$
        and $\forall a_i$ if $a_i = 0$ then $a_{i+1} \neq 0 $.
    Let $F(n) =$ the nth Fibonacci number.\\
    \section*{Proofs}
    \begin{theorem}
        let $g(x) = \sum_{i=0}^{n} a_i x^i$ with $a_n = a_0 = 1$  
        if $(x+1) * g(x) = 2^{n+1} -1$ 
        then $g(x)$ does not contain two consecutive terms with coefficient zero  
    
        \begin{proof}
            Assume for the sake of contradiction that there exists i 
            such that $0<j<n$,  $a_{j-1} = a_{j} = 0$ and $(x+1)*g(x) =f_{n+1}(x)$.
            Consider the following:
            \begin{align*}
                (x+1)*g(x)  &= x*g(x)+ 1*g(x)\\
                            &= x*\sum_{i=0}^{n} a_i x^i+ 1*\sum_{i=0}^{n} a_i x^i 
                                    &\text{(substituting in g(x))}\\
                            &= \sum_{i=0}^{n} a_i x^{i+1}+ \sum_{i=0}^{n} a_i x^i
                                    &\text{(multiplying by x in summation)}\\
                            &= a_nx^{n+1}+\sum_{i=0}^{n-1} a_i x^{i+1}+ \sum_{i=1}^{n} a_i x^i+ a_0
                                    &\text{(pulling 0th and (n+1)th terms out of the summation)}\\
                            &= a_nx^{n+1}+\sum_{i=1}^{n} a_{i-1} x^{i}+ \sum_{i=1}^{n} a_i x^i+ a_0
                                    &\text{(reindexing the first sum $i \leftarrow i+1$)}\\
                            &=a_nx^{n+1}+\sum_{i=1}^{n} (a_{i-1} + a_i) x^i+ a_0
                                    &\text{(combining summations)}\\
            \end{align*}
            by the assumption there exist j such that $a_{j-1} + a_{j} = 0$ 
            contradicting the assumption that $(x+1)*g(x) =f_{n+1}(x)$ because the jth term of $(x+1)*g(x)$'s coefficient  is 0  
        \end{proof}
    \end{theorem}

    
    

    
    \begin{theorem}
        $|G_n|= F(n+2)$ 
        \begin{proof}
            The proof will be by strong induction.
            \subsubsection*{n = 1}
                \begin{enumerate}
                    \item $g(x) = 1x$
                    \item $g(x) = 0x$
                \end{enumerate}
            \subsubsection*{n = 2}
                \begin{enumerate}
                    \item $g(x) =  1x^2 + x $
                    \item $g(x) =  0x^2 + x $
                    \item $g(x) = 1x^2 + 0x$
                \end{enumerate}
            The base case holds. 
        \subsubsection*{Induction hypothesis:}
                Assume that $|G_j| = F(j+2)$ for all $j \in \{1,2, \cdots, k\} $ for some k.
        \section*{Inductive step:}
                consider a $g(x) \in G_{k+1}$.
                \begin{enumerate}[label= \textbf{Case \arabic*.}]
                    \item $a_{k+1}=1.$ In which case the rest of the polynomial can be any polynomial in $G_k$
                    \item  $a_{k+1}=0.$ which implies the $a_k=1$. In which case the rest of the polynomial can be any polynomial in $G_{k-1}$.
                \end{enumerate}
                Since every polynomial is in $G_{k+1}$ one of these cases $|G_{k+1}| = |G_{k}| + |G_{k-1}| = F(k+2) + F(k+1)$.
                And by applying the Induction hypothesis $|G_k| = F(k+2) + F(k+1)= F(k+3)$.
            Therefore by the Principle of Mathematical Induction $|G_n| = F(n+2)$

        \end{proof}
        
    \end{theorem}
    
    \begin{theorem}
        let $g(x)\in G_{n-1}$. $ (x+1)(x^n +g(x) + 1) = f_{n+1}(x)$
        \begin{proof}
            let $g(x) = \sum_{i=0}^{n-1} a_i x^i$. such that if $a_i = 0$ then $a_{i+1} \neq 0 $
            consider the following
            \begin{align*}
                (x+1)(x^n +g(x) + 1)
                    &=(x+1)x^n +(x+1)g(x) + (x+1)\\
                    &= x^{n+1} + x^n +x(g(x))+ g(x)+ x+1\\
                    &= x^{n+1} + x^n +\sum_{i=1}^{n-1} a_i x^{i+1}+ \sum_{i=1}^{n-1} a_i x^i+ x+1 \\
                    &= x^{n+1} + x^n +\sum_{i=2}^{n} a_{i-1} x^{i}+ \sum_{i=1}^{n-1} a_i x^i+ x+1 &\text{(reindexing 1st sum $i\rightarrow i+1$)}\\
                    &= x^{n+1} + x^n +a_{n-1}x^n +\sum_{i=2}^{n-1} a_{i-1} x^{i} + \sum_{i=2}^{n-1} a_i x^i+ a_i x+ x+1 &\text{(matching sum indices)}\\
                    &= x^{n+1} + x^n + \sum_{i=2}^{n-1} (a_{i-1}+a_i) x^i+ x+1 & \text{(combining sums and simplifying)}\\
                    &= x^{n+1} + x^n + \sum_{i=2}^{n-1}  x^i+ x+1 & \text{(Since $(a_{i-1}+a_i)= 1)$}\\
                    &= f_{n+1}(x) & \text{(by definition of $f_{n+1}(x)$) } 
            \end{align*}
            Therefore $(x+1)(x^n +g(x) + 1) = f_{n+1}(x)$.
        \end{proof}
    \end{theorem}
\end{document}